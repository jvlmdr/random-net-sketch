\section{Network definition}

Consider a fully-connected network with $L$ linear layers.
Let $\obj{y}_{i}$ denote the pre-activations and $\obj{x}_{i}$ denote the output of the layer.
The network then maps $\obj{x}_{0} \mapsto \obj{y}_{L}$ according to:
\begin{equation}
\left\{ \begin{aligned}
\obj{y}_{i} & = \obj{W}_{i} \obj{x}_{i - 1} + \obj{b}_{i} && i = 1, \dots, L\\
\obj{x}_{i} & = \sigma(\obj{y}_{i}) && i = 1, \dots, L - 1
\end{aligned} \right.
\end{equation}
where $\sigma$ is the ReLU function $\sigma(y) = \max(0, y)$, and it is applied element-wise to a vector.
Let $m_{i}$ denote the dimension of the vectors $\obj{y}_{i}$ and $\obj{x}_{i}$.
Let the dimension of each matrix $W_{i}$ be denoted $m_{i} \times n_{i}$.
For a fully-connected network, we have $n_{i} = m_{i-1}$ (this is not the case for a convolutional network, where multiple pixels from the previous layer are used to compute one pixel in the current layer).

The loss is a scalar $\ell = E(\obj{y}_{L}) = f(\obj{x}_{0}, \obj{W}_{1}, \obj{b}_{1}, \dots, \obj{W}_{L}, \obj{b}_{L})$.
Let the notation $\gradobj{v} \equiv \partial \ell / \partial \obj{v}$ represent the derivative of the scalar $\ell$ with respect to any variable $\obj{v}$.
The dimension of $\gradobj{v}$ is identical to that of $\obj{v}$.
Gradient descent involves computing $\gradobj{W}_{i}$ and $\gradobj{b}_{i}$ for all $i$, which are obtained through back-propagation (see Section~\ref{sec:simple_derivatives} for the details of the derivation)
\begin{equation}
\left\{ \begin{aligned}
\gradobj{W}_{i} & = \gradobj{y}_{i} \obj{x}_{i - 1}^T & i & = L, \dots, 1 \\
\gradobj{b}_{i} & = \gradobj{y}_{i} & i & = L, \dots, 1 \\
\gradobj{x}_{i} & = \obj{W}_{i + 1}^T \gradobj{y}_{i + 1} & i & = L-1, \dots, 1 \\
\gradobj{y}_{i} & = \gradobj{x}_{i} \odot \nabla\sigma(\obj{y}_{i}) & i & = L - 1, \dots, 1
\end{aligned} \right.
\end{equation}
where $\odot$ denotes element-wise multiplication.
The pseudo-derivative of the ReLU function is $\nabla\sigma(y) = 1[y \ge 0]$.

\section{Random variables}

Let us consider all variables in the network $\obj{x}_{i}$, $\obj{y}_{i}$, $\obj{W}_{i}$, $\obj{b}_{i}$ to be random vectors with iid elements.
Since all elements $(\obj{v})_{u}$ of a vector $\obj{v}$ of iid random variables have the same distribution, let us use simply $v$ to denote a random variable with this distribution.

We can then study the distributions of $\grad{W}_{i}$ as a function of the input variables $x_{0}$, the parameters $W_{i}$ and $b_{i}$ and the gradient of the loss with respect to the output $\grad{y}_{L}$.
This will enable us to design a random network initialization with properties that we want.

%If we are not careful with the network initialization, we may have enormous differences in magnitude between the gradients and the weights.
%Is this a problem, if each layer has similar magnitude gradients and weights?

%Assume that each element of $\obj{x}_{0}$ is iid $\mathcal{N}(0, 1)$.

%Let us consider a random input $\obj{x}_{0}$ and random initial weights $\obj{W}_{i}$ and biases $\obj{b}_{i}$.
%What are the distributions of activations and gradients?

\section{Desiderata}

We desire the magnitude of the gradients relative to the parameters $\V[\grad{W}_{i}] / \V[W_{i}]$ to be roughly equal for all layers $i$.
This ensures that \emph{additive} gradient steps have a similar effect on the \emph{relative} change in each layer.
We might require the same of the bias parameters $\V[\grad{b}_{i}] / \V[b_{i}]$, except it is often useful in analysis to assume that $\obj{b}_{i} = 0$.
As an alternative, we might ask that the gradient with respect to the pre-activations $\V[\grad{b}_{i}] / \V[y_{i}]$ is similar for all layers $i$.
Note that if we can keep the magnitudes $\V[W_{i}]$ (resp.\ $\V[y_{i}]$) roughly equal, then it is sufficient to ensure that the gradients $\V[\grad{W}_{i}]$ (resp.\ $\V[\grad{b}_{i}]$) are roughly equal across layers.

Furthermore, we should choose $\V[W_{i}]$ to be roughly equal for all layers $i$.
This is important because gradient descent with learning rate $\eta$ effectively optimizes a first-order approximation with Hessian $(1/\eta) I$:
\begin{equation}
\theta_{t+1} = \theta_{t} - \eta g_{t}
= \arg \min_{\theta} \{ f(\theta_{t}) + g_{t}^{T} (\theta - \theta_{t}) + \tfrac{1}{2 \eta} \| \theta - \theta_{t} \|^{2} \}
\end{equation}
where $g_{t} = \nabla_{\theta} f(\theta_{t})$.
This encodes a prior that each step $\theta - \theta_{t}$ should have roughly equal magnitude in all directions.

It is important that the output of this random network is in a reasonable range, such as $\E[y_{L}] \approx 0$ and $\V[y_{L}] \approx 1$.

Finally, the relative magnitude of the gradients should be similar for different networks, such that the optimization procedure is not too sensitive to the choice of learning rate.
The magnitude of the gradients for the weights relative to the gradients for the biases similarly should not vary much between networks.

\section{Analysis}

Let us assume that $\E[x_{0}] = 0$ and $\V[x_{0}] = 1$.
If we further assume that $\obj{b}_{i} = 0$, that the distribution of $y_{i}$ is symmetric around zero, and that gradients and values are independent, then we obtain (see Section~\ref{} for detailed derivation):
\begin{align}
%\E[x_{i}^2] & = \tfrac{1}{2} \V[y_{i}] = \tfrac{1}{2} m_{i - 1} \V[W_{i}] \E[x_{i - 1}^2] && i = 1, \dots, L \\
\V[y_{i}] & = m_{i - 1} \V[W_{i}] \E[x_{i - 1}^2] && i = 1, \dots, L \\
\E[x_{i}^2] & = \tfrac{1}{2} \V[y_{i}] \nonumber \\
& = \tfrac{1}{2} m_{i - 1} \V[W_{i}] \E[x_{i - 1}^2] && i = 1, \dots, L-1 \\
\V[\grad{x}_{i}] & = m_{i+1} \V[W_{i+1}] \V[\grad{y}_{i+1}] && i = 0, \dots, L-1 \\
\V[\grad{y}_{i}] & = \tfrac{1}{2} \V[\grad{x}_{i}] \nonumber \\
& = \tfrac{1}{2} m_{i+1} \V[W_{i+1}] \V[\grad{y}_{i+1}] && i = 1, \dots, L-1 \\
\V[\grad{W}_{i}] & = \V[\grad{y}_{i}] \E[x_{i-1}^2] && i = 1, \dots, L
\end{align}
If we expand these recursions, we obtain
\begin{align}
\E[x_{i}^2] & = \left( \prod_{j = 1}^{i} \tfrac{1}{2} m_{j - 1} \V[W_{j}] \right) \E[x_{0}^2] \\
\V[\grad{y}_{i}] & = \left( \prod_{j = i+1}^{L} \tfrac{1}{2} m_{j} \V[W_{j}] \right) \V[\grad{y}_{L}] \\
\V[\grad{W}_{i}] & =
  \left( \prod_{j = 1}^{i-1} \tfrac{1}{2} m_{j - 1} \V[W_{j}] \right)
  \left( \prod_{j = i+1}^{L} \tfrac{1}{2} m_{j} \V[W_{j}] \right)
  \E[x_{0}^2] \V[\grad{y}_{L}]
\end{align}
and recall that $\V[\grad{b}_{i}] = \V[\grad{y}_{i}]$.
Furthermore, if we introduce a constant
\begin{equation}
C = \frac{1}{2^{L-1}} \left(\prod_{j=0}^{L} m_{j} \right) \left(\prod_{j=1}^{L} \V[W_{j}] \right)
  \E[x_{0}^2] \V[\grad{y}_{L}]
\end{equation}
then we can express the above as
\begin{equation}
\V[\grad{W}_{i}] = \frac{C}{m_{i-1} m_{i} \V[W_{i}]}
\end{equation}
In fact, this shows that the gradient is inversely proportional to the magnitude of the weights $\V[\grad{W}_{i}] \propto \V[W_{i}]^{-1}$.
This is the opposite of the desired situation $\V[\grad{W}_{i}] \propto \V[W_{i}]$.
However, if we choose all $\V[W_{i}] \approx \V[W_{j}]$ and $m_{i} \approx m_{j}$, we will also have gradients of similar magnitudes $\V[\grad{W}_{i}] \approx \V[\grad{W}_{j}]$.

\section{Achieving the desiderata}

%Let $\alpha_{i}^2 = \V[W_{i}]$ denote the magnitude of the random weights at each layer.
This section will aim to choose values for $\V[W_{i}]$ which achieve the desiderata as best possible.

It is evident above that the magnitudes of the gradients $\V[\grad{W}_{i}]$ are approximately equal for all layers, assuming that the values of $m_{i-1}$, $m_{i}$ and $\V[W_{i}]$ do not vary too much.
However, if we inspect the gradients for the biases
\begin{equation}
\V[\grad{b}_{i}] = \V[\grad{y}_{i}] =\tfrac{1}{2} m_{i+1} \V[W_{i+1}] \V[\grad{y}_{i+1}] = \left( \prod_{j = i+1}^{L} \tfrac{1}{2} m_{j} \V[W_{j}] \right) \V[\grad{y}_{L}]
\end{equation}
we can see that these may grow rapidly.
To avoid this, we can simply choose $\V[W_{i}] = 2 / m_{i}$ for $i = 2, \dots, L$.
(Note that there is no constraint on $\V[W_{1}]$.)

In the PReLU paper, they suggest as a desiderata that the magnitude of the activations at different layers are kept constant.
Examining the variance of the activations (defining $\V[y_{0}] = 2 \E[x_{0}^2]$ for convenience)
\begin{align}
\V[y_{i}] & = \tfrac{1}{2} m_{i - 1} \V[W_{i}] \V[y_{i - 1}^2] && i = 1, \dots, L
\end{align}
we see that this can be achieved by setting $\V[W_{i}] = 2 / m_{i-1}$ for $i = 1, \dots, L$.
This solution satisfies another of our desiderata: it ensures that the variance of the output $\V[y_{L}]$ is in a reasonable range
\begin{equation}
\V[y_{L}] = 2 \left( \prod_{j = 1}^{L} \tfrac{1}{2} m_{j - 1} \V[W_{j}] \right) \E[x_{0}^2] = 2 \E[x_{0}^2]
\end{equation}

\section{Relative gradients}

For the weights:
\begin{align}
\frac{\V[\grad{W}_{i}]}{\V[W_{i}]}
& = \frac{\V[\grad{y}_{i}] \E[x_{i-1}^2]}{\V[W_{i}]} \\
& = \frac{1}{\V[W_{i}]} \left( \tfrac{1}{2} m_{i+1} \V[W_{i+1}] \V[\grad{y}_{i+1}] \right) \left( \frac{\E[x_{i}^2]}{\tfrac{1}{2} m_{i - 1} \V[W_{i}]} \right) \\
%& = \frac{m_{i+1} \V[W_{i+1}] \V[\grad{y}_{i+1}] \E[x_{i}^2]}{m_{i - 1} \V[W_{i}]^2} \\
& = \frac{m_{i+1} \V[W_{i+1}]}{m_{i - 1} \V[W_{i}]^2} \V[\grad{y}_{i+1}] \E[x_{i}^2] \\
& = \left(\frac{m_{i+1} \V[W_{i+1}]^2}{m_{i - 1} \V[W_{i}]^2}\right) \frac{\V[\grad{W}_{i+1}]}{\V[W_{i+1}]}
%& = \left( \frac{\sqrt{m_{i+1}} \V[W_{i+1}]}{\sqrt{m_{i - 1}} \V[W_{i}]} \right)^2 \frac{\V[\grad{W}_{i+1}]}{\V[W_{i+1}]}
\end{align}
or, using the global constant $C$, simply:
\begin{align}
\frac{\V[\grad{W}_{i}]}{\V[W_{i}]} = \frac{C}{m_{i-1} m_{i} \V[W_{i}]^2}
\end{align}
For the activations:
\begin{align}
\frac{\V[\grad{y}_{i}]}{\V[y_{i}]}
& = \frac{
  \tfrac{1}{2} m_{i+1} \V[W_{i+1}] \V[\grad{y}_{i+1}]}{
  \V[y_{i+1}] / \left( \tfrac{1}{2} m_{i} \V[W_{i+1}] \right)}
%& = \frac{m_{i} m_{i+1} \V[W_{i+1}]^2}{2^2} \frac{\V[\grad{y}_{i+1}]}{\V[y_{i+1}]} \\
= \left(\frac{\sqrt{m_{i} m_{i+1}}}{2} \V[W_{i+1}]\right)^2 \frac{\V[\grad{y}_{i+1}]}{\V[y_{i+1}]}
\end{align}

\section{Desiderata for parameter gradients}

Alternatively, we might choose $\V[W_{i}]$ to effect equality between the magnitudes of the gradients $\V[\grad{W}_{i}]$.
On examination of the variance
\begin{equation}
\V[\grad{W}_{i}] = \frac{C}{m_{i-1} m_{i} \V[W_{i}]}
\end{equation}
we see that this could be achieved by taking $\V[W_{i}] = 1 / (m_{i-1} m_{i})$.
However, recall that we are more interested in the relative magnitude
\begin{equation}
\frac{\V[\grad{W}_{i}]}{\V[W_{i}]} = \frac{C}{m_{i-1} m_{i} \V[W_{i}]^2}
\end{equation}
In this case, we should set $\V[W_{i}] = 1 / \sqrt{m_{i-1} m_{i}}$ for $i = 1, \dots, L$.
This is also an elegant middle ground between the measures outlined above.

Note that there is an additional degree of freedom here: we are free to choose a global scale $\alpha^2$ such that $\V[W_{i}] = \alpha^2 / \sqrt{m_{i-1} m_{i}}$.
This can alternatively be seen by deriving a relative constraint
\begin{align}
\frac{\V[\grad{W}_{i}]}{\V[W_{i}]} = \frac{C}{m_{i-1} m_{i} \V[W_{i}]^2}
  & = \frac{C}{m_{i} m_{i+1} \V[W_{i+1}]^2} = \frac{\V[\grad{W}_{i+1}]}{\V[W_{i+1}]} \\
%\frac{1}{m_{i-1} \V[W_{i}]^2} & = \frac{1}{m_{i+1} \V[W_{i+1}]^2} \\
\V[W_{i+1}]^2 & = \frac{m_{i-1}}{m_{i+1}} \V[W_{i}]^2
%\V[W_{i+1}] & = \frac{\sqrt{m_{i-1}}}{\sqrt{m_{i+1}}} \V[W_{i}]
\end{align}
which is satisfied by $\V[W_{i}] = \alpha^2 / \sqrt{m_{i-1} m_{i}}$.

Now let's examine the ratio $\rho_{i} = \V[\grad{W}_{i}] / \V[W_{i}]$ for the case $\V[W_{i}] = \alpha^{2} / \sqrt{m_{i-1} m_{i}}$.
\begin{align}
\frac{\V[\grad{W}_{i}]}{\V[W_{i}]} = \frac{C}{(\alpha^2)^2}
& = \frac{1}{(\alpha^2)^2} \frac{1}{2^{L-1}} \left(\prod_{j=0}^{L} m_{j} \right) \left(\prod_{j=1}^{L} \frac{\alpha^{2}}{\sqrt{m_{i-1} m_{i}}} \right) \E[x_{0}^2] \V[\grad{y}_{L}] \\
& = \frac{(\alpha^2)^{L-2}}{2^{L-1}} m_{0} \left(\prod_{j=1}^{L} \frac{\sqrt{m_{j}}}{\sqrt{m_{j-1}}} \right) \E[x_{0}^2] \V[\grad{y}_{L}] \\
%& = \frac{(\alpha^2)^{L-2}}{2^{L-1}} m_{0} \frac{\sqrt{m_{L}}}{\sqrt{m_{0}}} \E[x_{0}^2] \V[\grad{y}_{L}] \\
& = \left(\frac{\alpha^2}{2}\right)^{L-1} \frac{\sqrt{m_{0} m_{L}}}{\alpha^2} \E[x_{0}^2] \V[\grad{y}_{L}]
\end{align}
What about the ratio $\V[\grad{b}_{i}] / \V[y_{i}] = \V[\grad{y}_{i}] / \V[y_{i}]$?
\begin{align}
\frac{\V[\grad{y}_{i}]}{\V[y_{i}]} & = \left.
  \left[ \left( \prod_{j = i+1}^{L} \tfrac{1}{2} m_{j} \V[W_{j}] \right) \V[\grad{y}_{L}] \right] \middle/
  \left[ 2 \left( \prod_{j = 1}^{i} \tfrac{1}{2} m_{j - 1} \V[W_{j}] \right) \E[x_{0}^2] \right] \right. \\
& = \frac{1}{2} \frac{1}{2^{L - 2 i}} \frac{\V[\grad{y}_{L}]}{\E[x_{0}^2]} \left.
  \left( \prod_{j = i+1}^{L} \alpha^2 \frac{\sqrt{m_{j}}}{\sqrt{m_{j-1}}} \right) \middle/
  \left( \prod_{j = 1}^{i} \alpha^2 \frac{\sqrt{m_{j-1}}}{\sqrt{m_{j}}} \right) \right. \\
& = \frac{1}{2} \left(\frac{\alpha^2}{2}\right)^{L - 2 i} \sqrt{\frac{m_{L}}{m_{0}}} \frac{\V[\grad{y}_{L}]}{\E[x_{0}^2]}
\end{align}
This depends on the depth $i$ with the variance changing by a factor of $(\alpha^2 / 2)^2$ per layer.
This suggests that we can choose $\alpha^2 = 2$, or $\alpha_{i}^2 = 2 / \sqrt{m_{i-1} m_{i}}$.
(This can also be seen more easily in the original equation for $\V[\grad{y}_{i}] / \V[y_{i}]$.)
Then we have simply
\begin{align}
\frac{\V[\grad{W}_{i}]}{\V[W_{i}]} & = \frac{1}{2} \sqrt{m_{0} m_{L}} \E[x_{0}^2] \V[\grad{y}_{L}] \\
\frac{\V[\grad{y}_{i}]}{\V[y_{i}]} & = \frac{1}{2} \frac{\sqrt{m_{L}}}{\sqrt{m_{0}}} \frac{\V[\grad{y}_{L}]}{\E[x_{0}^2]}
\end{align}
independent of the layer $i$.

%Is there some way to choose $\V[W_{i}]$ to achieve constant $\V[\grad{y}_{i}] / \V[y_{i}]$?
%\begin{align}
%\frac{\V[\grad{y}_{i}]}{\V[y_{i}]} = \frac{\V[\grad{y}_{L}]}{2 \E[x_{0}^2]} \left.
%  \left( \prod_{j = i+1}^{L} \tfrac{1}{2} m_{j} \V[W_{j}] \right) \middle/
%  \left( \prod_{j = 1}^{i} \tfrac{1}{2} m_{j - 1} \V[W_{j}] \right) \right.
%\end{align}
%This is difficult because each equation depends on almost all $\V[W_{i}]$.
%If we take the log of each equation, we obtain a system of $L$ equations and $L + 1$ variables including the ratio $r = \V[\grad{y}_{i}] / \V[y_{i}]$.
%\begin{multline}
%(L - 2 i + 1) \log 2 + \sum_{j = 1}^{i} \log m_{j-1} - \sum_{j = i+1}^{L} \log m_{j} \\
%= -\log r - \sum_{j = 1}^{i} \log \V[W_{j}] + \sum_{j = i+1}^{L} \log \V[W_{j}]
%\end{multline}
%%What extra constraint can we use to choose $r$?
%This is neat, but unfortunately it requires global knowledge of the network in order to obtain the initialization.
%Furthermore, it does not preserve the relative magnitude of the other gradients $\V[\grad{W}_{i}] / \V[W_{i}]$.

Finally, consider the \emph{ratio} of relative gradients for $\gradobj{W}_{i}$ to those for $\gradobj{y}_{i}$.
\begin{align}
\left. \frac{\V[\grad{W}_{i}]}{\V[W_{i}]} \middle/ \frac{\V[\grad{y}_{i}]}{\V[y_{i}]} \right.
& = \left.
  \left( \frac{1}{2} \sqrt{m_{0} m_{L}} \E[x_{0}^2] \V[\grad{y}_{L}] \right) \middle/
  \left( \frac{1}{2} \frac{\sqrt{m_{L}}}{\sqrt{m_{0}}} \frac{\V[\grad{y}_{L}]}{\E[x_{0}^2]} \right) \right. \\
& = m_{0} \E[x_{0}^2]^2
\end{align}
This shows that the dimension $m_{0}$ and variance $\E[x_{0}^2]$ affect the relative learning rate of the weights and biases.

If the base learning rate is $\lambda$, we could adopt parameter-specific learning rates $\lambda_{W}$ and $\lambda_{b}$ that satisfy
\begin{align}
\lambda_{W}^2 & = \frac{2}{\sqrt{m_{0} m_{L}}} \lambda^2 \\
\lambda_{b}^2 & = \frac{2 \sqrt{m_{0}}}{\sqrt{m_{L}}} \lambda^2
\end{align}
and hence $\lambda_{W} / \lambda_{b} = 1 / \sqrt{m_{0}}$ and the scaled gradients should be roughly equal
\begin{align}
\lambda_{W} \sqrt{\frac{\V[\grad{W}_{i}]}{\V[W_{i}]}} & = \lambda_{b} \sqrt{\frac{\V[\grad{b}_{i}]}{\V[y_{i}]}}
\end{align}

\section{}

Let's briefly examine the general case with arbitrary $\V[W_{i}]$
\begin{align}
\frac{ \V[\grad{W}_{i}] / \V[W_{i}] }{ \V[\grad{y}_{i}] / \V[y_{i}] }
& = \frac{
  \left(\frac{\sqrt{m_{i+1}} \V[W_{i+1}]}{\sqrt{m_{i - 1}} \V[W_{i}]}\right)^2 \frac{\V[\grad{W}_{i+1}]}{\V[W_{i+1}]} }{
  \left(\frac{\sqrt{m_{i} m_{i+1}}}{2} \V[W_{i+1}]\right)^2 \frac{\V[\grad{y}_{i+1}]}{\V[y_{i+1}]} } \\
& = \left( \frac{\sqrt{m_{i+1}} \V[W_{i+1}]}{\sqrt{m_{i - 1}} \V[W_{i}]}
  \frac{2}{\sqrt{m_{i} m_{i+1}} \V[W_{i+1}]} \right)^2
  \frac{ \V[\grad{W}_{i+1}] / \V[W_{i+1}] }{ \V[\grad{y}_{i+1}] / \V[y_{i+1}] } \\
& = \left( \frac{2}{\sqrt{m_{i} m_{i - 1}} \V[W_{i}]} \right)^2
  \frac{ \V[\grad{W}_{i+1}] / \V[W_{i+1}] }{ \V[\grad{y}_{i+1}] / \V[y_{i+1}] }
\end{align}
Of course this quantity can be made constant for all $i$ using $\V[W_{i}] = 2 / \sqrt{m_{i} m_{i-1}}$.


\section{Re-parameterization}

Now let's consider an alternative network that introduces a constant scalar $a_{i}$ at each layer and re-parameterizes $\obj{W}_{i} = a_{i} \obj{U}_{i}$
\begin{equation}
\left\{ \begin{aligned}
\obj{y}_{i} & = a_{i} \obj{U}_{i} \obj{x}_{i - 1} + \obj{b}_{i} && i = 1, \dots, L\\
\obj{x}_{i} & = \sigma(\obj{y}_{i}) && i = 1, \dots, L - 1
\end{aligned} \right.
\end{equation}
and the derivatives therefore become
\begin{equation}
\left\{ \begin{aligned}
\gradobj{U}_{i} & = a_{i} \gradobj{y}_{i} \obj{x}_{i - 1}^T && i = L, \dots, 1 \\
\gradobj{b}_{i} & = \gradobj{y}_{i} && i = L, \dots, 1 \\
\gradobj{x}_{i} & = a_{i+1} \obj{U}_{i + 1}^T \gradobj{y}_{i + 1} = \obj{W}_{i + 1}^T \gradobj{y}_{i + 1} && i = L - 1, \dots, 1 \\
\gradobj{y}_{i} & = \nabla\sigma(\obj{y}_{i}) \odot \gradobj{x}_{i} && i = L - 1, \dots, 1
\end{aligned} \right.
\end{equation}

Now let us compare the relative magnitudes of the gradients:
\begin{align}
\gradobj{U}_{i} & = a_{i} \gradobj{W}_{i} \\
\obj{U}_{i} & = \obj{W}_{i} / a_{i} \\
\frac{\| \gradobj{U}_{i} \|}{\| \obj{U}_{i} \|} & = a_{i}^{2} \frac{\| \gradobj{W}_{i} \|}{\| \obj{W}_{i} \|}
\end{align}
(for any norm).
Hence this re-parameterization can have an enormous effect on optimization.

In fact, we can show this much more simply.
Let $y = f(x)$.
Now let $u = x / a$ for some constant $a$ and consider $y = f(a u)$.
The derivative is $\partial y / \partial u = a \nabla f(a u) = a \nabla f(x)$ and hence
\begin{equation}
\frac{\partial y / \partial u}{u} = \frac{a \nabla f(x)}{x / a} = a^2 \frac{\partial y / \partial x}{x}
\end{equation}
Why not introduce such scalars in the network initialization to achieve the desired gradients?

%Compared to the original network, we have $\V[W_{i}] = a_{i}^2 \V[U_{i}] = a_{i}^2 s_{i}^2$.
Whereas for the original network, we have
\begin{align}
\V[y_{i}] & = m_{i - 1} \V[W_{i}] \E[x_{i - 1}^2] \\
\E[x_{i}^2] & = \tfrac{1}{2} \V[y_{i}] \\
\V[\grad{x}_{i}] & = m_{i+1} \V[W_{i+1}] \V[\grad{y}_{i+1}] \\
\V[\grad{y}_{i}] & = \tfrac{1}{2} \V[\grad{x}_{i}] \\
\V[\grad{W}_{i}] & = \V[\grad{y}_{i}] \E[x_{i - 1}^2]
\end{align}
with the re-parameterization $W_{i} = a_{i} U_{i}$, we instead have
\begin{align}
\V[y_{i}] & = m_{i-1} a_{i}^2 \V[U_{i}] \E[x_{i-1}^2] \\
\E[x_{i}^2] & = \tfrac{1}{2} \V[y_{i}] \\
\V[\grad{x}_{i}] & = m_{i+1} a_{i+1}^2 \V[U_{i+1}] \V[\grad{y}_{i+1}] \\
\V[\grad{y}_{i}] & = \tfrac{1}{2} \V[\grad{x}_{i}] \\
\V[\grad{U}_{i}] & = a_{i}^2 \V[\grad{y}_{i}] \E[x_{i-1}^2]
\end{align}
(with the assumption that $\obj{b}_{i} = 0$).
The equations for $\V[y_{i}]$ and $\V[\grad{x}_{i}]$ incorporate the new variance $\V[W_{i}] = a_{i}^2 \V[U_{i}]$, and the equation for $\V[\grad{U}_{i}]$ also incorporates the scale factor $a_{i}$.

The key difference is in the relative magnitude
\begin{align}
\frac{\V[\grad{U}_{i+1}]}{\V[\grad{U}_{i}]}
= \frac {a_{i}^2 \V[\grad{y}_{i}] \E[x_{i-1}^2]} {\V[U_{i}]}
= \frac {a_{i}^2 \V[\grad{y}_{i}] \E[x_{i-1}^2]} {(1/a_{i}^2) \V[W_{i}]}
= a_{i}^4 \frac{\V[\grad{W}_{i}]}{\V[W_{i}]}
\end{align}
Therefore the parameters $a_{i}$ provide a mechanism to control the relative scale of gradients and weights.
This might be more or less equivalent to a parameter-specific learning rate?


\begin{subappendices}

\section{Derivatives for simple network}
\label{sec:simple_derivatives}
The differentials of the equations that define the network are:
\begin{equation}
\left\{ \begin{aligned}
\obj{dx}_{i} & = \nabla\sigma(\obj{y}_{i}) \odot \obj{dy}_{i} \\
\obj{dy}_{i} & = \obj{dW}_{i} \obj{x}_{i - 1} + \obj{W}_{i} dx_{i - 1} + \obj{db}_{i}
\end{aligned} \right.
\end{equation}

For the first equation:
\begin{align}
\langle \gradobj{y}_{i}, \obj{dy}_{i} \rangle & = \langle \gradobj{y}_{i}, \obj{dW}_{i} \obj{x}_{i - 1} + \obj{W}_{i} \obj{dx}_{i - 1} + \obj{db}_{i} \rangle \\
& = \langle \gradobj{y}_{i} \obj{x}_{i - 1}^T, \obj{dW}_{i} \rangle + \langle \obj{W}_{i}^T \gradobj{y}_{i}, \obj{dx}_{i - 1} \rangle + \langle \gradobj{y}_{i}, \obj{db}_{i} \rangle \\
& = \langle \gradobj{W}_{i}, \obj{dW}_{i} \rangle + \langle \gradobj{x}_{i-1}, \obj{dx}_{i-1} \rangle + \langle \gradobj{b}_{i}, \obj{db}_{i} \rangle
\end{align}
This provides the derivatives:
\begin{align}
\gradobj{W}_{i} & = \gradobj{y}_{i} \obj{x}_{i - 1}^T \\
\gradobj{x}_{i - 1} & = \obj{W}_{i}^T \gradobj{y}_{i} \\
\gradobj{b}_{i} & = \gradobj{y}_{i}
\end{align}

For the second equation:
\begin{align}
\langle \gradobj{x}_{i}, dx_{i} \rangle & = \langle \gradobj{x}_{i}, \nabla\sigma(\obj{y}_{i}) \odot \obj{dy}_{i} \rangle \\
& = \langle \nabla\sigma(\obj{y}_{i}) \odot \gradobj{x}_{i}, \obj{dy}_{i} \rangle \\
\gradobj{y}_{i} & = \nabla\sigma(\obj{y}_{i}) \odot \gradobj{x}_{i}
\end{align}


\section{Analysis of simple network}
\label{sec:simple_analysis}
This analysis follows that of the PReLU paper.
\begin{align}
(\obj{y}_{i})_{u} & = \sum_{v = 1}^{n_{i}} (\obj{W}_{i})_{u v} (\obj{x}_{i - 1})_{v} + (\obj{b}_{i})_{v} \\
\V[y_{i}] & = n_{i} \V[W_{i} \cdot x_{i-1}] + \V[b_{i}] \\
\intertext{then, using the independence of $W_{i}$ and $x_{i-1}$ and that $\E[W_{i}] = 0$}
\V[y_{i}] & = n_{i} \V[W_{i}] \E[x_{i-1}^2] + \V[b_{i}]
\end{align}
We can also easily observe that $\E[y_{i}] = \E[b_{i}]$.

To find the variance of $\obj{x}_{i} = \sigma(\obj{y}_{i})$, it helps to assume that $\obj{b}_{i} = 0$ and therefore $\E[y_{i}] = 0$.
Let us further assume that the distribution of $y_{i}$ is symmetric about zero.
Then we can obtain
\begin{align}
\E[x_{i}^2] = \E[\sigma^2(y_{i})] = \int_{0}^{\infty} y_{i}^2 dP(y_{i}) = \frac{1}{2} \int_{-\infty}^{\infty} y_{i}^2 dP(y_{i}) = \frac{1}{2} \V[y_{i}]
\end{align}

As an aside, if we are interested in the variance $\V[x_{i}]$ rather than the squared magnitude $\E[x_{i}^2]$, then we can explicitly assume that the distribution of $y_{i}$ is Gaussian (which is plausible since it is a sum of many random variables and may therefore be approximately Gaussian according to the Central Limit Theorem), and obtain
\begin{align}
\E[x_{i}] & = \int_{0}^{\infty} y_{i} dP(y_{i}) = \tfrac{1}{\sqrt{2 \pi}} \sqrt{\V[y_{i}]} \approx 0.399 \sqrt{\V[y_{i}]} \\
\V[x_{i}] & = \E[x_{i}^2] - \E[x_{i}]^2 = (\tfrac{1}{2} - \tfrac{1}{2 \pi}) \V[y_{i}] \approx 0.341 \V[y_{i}]
\end{align}

This gives us a recursion between $\E[x_{i}^2]$ and $\E[x_{i - 1}^2]$
\begin{align}
\E[x_{i}^2] = \tfrac{1}{2} \V[y_{i}] & = \tfrac{1}{2} m_{i - 1} \V[W_{i}] \E[x_{i - 1}^2] \\
& = \left(\tfrac{1}{2} m_{i - 1} \V[W_{i}]\right) \left(\tfrac{1}{2} m_{i-2} \V[W_{i-1}]\right) \E[x_{i-2}^2] \\
& = \left(\tfrac{1}{2} m_{i - 1} \V[W_{i}]\right) \dots \left(\tfrac{1}{2} m_{0} \V[W_{1}]\right) \E[x_{0}^2] \\
& = \left( \prod_{j = 1}^{i} \tfrac{1}{2} m_{j - 1} \V[W_{j}] \right) \E[x_{0}^2]
\end{align}

Now let us consider the gradients.
First we establish that $\E[\grad{x}_{i}] = \E[\grad{y}_{i}] = 0$.
\begin{align}
(\gradobj{x}_{i})_{u} & = (\obj{W}_{i+1}^{T} \gradobj{y}_{i+1})_{u} = \sum_{v = 1}^{m_{i+1}} (\obj{W}_{i+1})_{v u} (\gradobj{y}_{i+1})_{v} \\
\E[\grad{x}_{i}] & = m_{i+1} \E[W_{i+1} \grad{y}_{i+1}] \\
\intertext{and if we then assume that $\grad{y}_{i+1}$ is independent of $W_{i+1}$, then we obtain}
& = m_{i+1} \E[W_{i+1}] \E[\grad{y}_{i+1}] = 0
\end{align}
We can also observe that
\begin{align}
\E[\grad{y}_{i}] & = \E[\nabla\sigma(y_{i}) \cdot \grad{x}_{i}] \\
\intertext{using the independence of $y_{i}$ and $\grad{x}_{i}$, and the symmetry of $y_{i}$ about zero}
& =  \E[\nabla\sigma(y_{i})] \E[\grad{x}_{i}] = \tfrac{1}{2} \E[\grad{x}_{i}] = 0
\end{align}
The variances can then be established
\begin{align}
\V[\grad{y}_{i}]
& = \V[\nabla \sigma(y_{i}) \cdot \grad{x}_{i}] \\
& = \E[(\nabla \sigma(y_{i}))^2] \E[\grad{x}_{i}^2] - \E[\nabla \sigma(y_{i})]^2 \E[\grad{x}_{i}]^2 \\
& = \tfrac{1}{2} \V[\grad{x}_{i}]
\end{align}
and
\begin{align}
(\gradobj{x}_{i})_{u} & = (\obj{W}_{i+1}^{T} \gradobj{y}_{i+1})_{u} = \sum_{v = 1}^{m_{i+1}} (\obj{W}_{i+1})_{v u} (\gradobj{y}_{i+1})_{v} \\
\V[\grad{x}_{i}] & = m_{i+1} \V[W_{i+1} \grad{y}_{i+1}] \\
& = m_{i+1} \left( \E[W_{i+1}^2] \E[\grad{y}_{i+1}^2] - \E[W_{i+1}]^2 \E[\grad{y}_{i+1}]^2 \right) \\
& = m_{i+1} \V[W_{i+1}] \V[\grad{y}_{i+1}]
\end{align}
The original gradient starts at $\gradobj{y}_{L}$.
\begin{align}
\V[\grad{y}_{i}] = \tfrac{1}{2} \V[\grad{x}_{i}] & = \tfrac{1}{2} m_{i+1} \V[W_{i+1}] \V[\grad{y}_{i+1}] \\
& = \left(\tfrac{1}{2} m_{i+1} \V[W_{i+1}]\right) \left(\tfrac{1}{2} m_{i+2} \V[W_{i+2}]\right) \V[\grad{y}_{i+2}] \\
& = \left(\tfrac{1}{2} m_{i+1} \V[W_{i+1}]\right) \cdots \left(\tfrac{1}{2} m_{L} \V[W_{L}]\right) \V[\grad{y}_{L}] \\
& = \left( \prod_{j = i+1}^{L} \tfrac{1}{2} m_{j} \V[W_{j}] \right) \V[\grad{y}_{L}]
\end{align}
Finally, we can inspect the gradients with respect to the parameters
\begin{align}
(\gradobj{W}_{i})_{u v} & = (\gradobj{y}_{i})_{u} (\obj{x}_{i - 1})_{v} \\
\V[\grad{W}_{i}] & = \V[\grad{y}_{i} x_{i-1}] \\
& = \E[\grad{y}_{i}^2] \E[x_{i-1}^2] - \E[\grad{y}_{i}]^2 \E[x_{i-1}]^2 \\
& = \V[\grad{y}_{i}] \E[x_{i-1}^2]
\end{align}
And if we substitute the results from above, we obtain
\begin{align}
\V[\grad{W}_{i}] & = \V[\grad{y}_{i}] \E[x_{i-1}^2] \\
& = \left( \prod_{j = 1}^{i-1} \tfrac{1}{2} m_{j - 1} \V[W_{j}] \right)
  \left( \prod_{j = i+1}^{L} \tfrac{1}{2} m_{j} \V[W_{j}] \right)
  \E[x_{0}^2] \V[\grad{y}_{L}]
\end{align}
If we introduce a constant
\begin{equation}
C = \frac{1}{2^{L-1}} \left(\prod_{j=0}^{L} m_{j} \right) \left(\prod_{j=1}^{L} \V[W_{j}] \right)
  \E[x_{0}^2] \V[\grad{y}_{L}]
\end{equation}
then we can express the above as
\begin{equation}
\V[\grad{W}_{i}] = \frac{C}{m_{i-1} m_{i} \V[W_{i}]}
\end{equation}
In fact, this shows that the relative magnitude $\V[\grad{W}_{i}] / \V[W_{i}]$ is quite close to a constant.

\end{subappendices}
